\section{Introduction}
\subsection{Computers and Real Numbers}

\begin{frame}
  \begin{itemize}
      \item Computers are often used to solve problems involving the real numbers
      \item Many applications in science and engineering
      \item The set of reals is uncountable
      \item Thus, it is not possible to represent all real numbers finitely
      \item \ldots and there are more real numbers than computer programs
      \item Classically, Computability and Complexity Theory deal only with finite structures (natural numbers, graphs,\ldots)
      \item How can we define computability of real numbers and functions?
  \end{itemize}
\end{frame}
\subsection{Computing on reals}

\begin{frame}
  For most real numbers, it is impossible to output them exactly in finite time.
  Instead a real number is called computable if it can be approximated up to any desired precision.
  \vfill
  \pause
\begin{minipage}{.45\textwidth}
		\begin{figure}
		\centering
    \vfill
		\begin{tikzpicture}<2->
				\path (0,0) rectangle (3.5,-2.7);
			%x
				\draw<2->[thick] (0,0) rectangle (3,-2);
				\node<2-> at (1.5,-1) {$x$};
				\draw<2-> (.5,-2) -- (.5,-2.2);
				\draw<2->[-triangle 90] (.5,-2.5) -- (.5,-2.2);
				\node<2-> at (.25,-2.25) {$1^n$};
				\draw<2->[-triangle 90] (2.5,-2) -- (2.5,-2.3);
				\draw<2-> (2.5,-2.3) -- (2.5,-2.5);
				\node<2-> at (2.75,-2.25) {$d_n$};
				%\node<4-> at (2.85,-2.25) {$d_{n,i}$};
				%\node<4-> at (1.5,-1) {$(a_i)$};
				%\draw<4-> (1,-2) -- (1,-2.2);
				%\draw<4->[-triangle 90] (1,-2.5) -- (1,-2.2);
				%\node<4-> at (1.25,-2.25) {$1^i$};
		\end{tikzpicture}
		\end{figure}
	\end{minipage}
	\hfill
	\begin{minipage}{.45\textwidth}
	\begin{definition}
		A real number $x$ is called computable if there is a computable function $d:\N\to\Z$, such that $ \left|\frac{d(n)}{2^{n+1}} - x\right|\leq 2^{-n}$.   
	\end{definition}
	\end{minipage}
\end{frame}
\begin{frame}
  \frametitle{Non-computable numbers}
  \begin{example}[Specker]
    Let $H \subseteq \N$ be any non-computable subset of the natural numbers (e.g. the Halting problem).
    \pause
    The real number defined by 
    $$
      \sum_{i \in H} 4^{-i}
    $$
    is not computable.
  \end{example}
\end{frame}
\subsection{Real functions}

\begin{frame}
	\begin{minipage}{.45\textwidth}
		\begin{tikzpicture}
			\path (-1.7,1.7) rectangle (3.7,-5.2);
			%x
				\draw<1-> (0,0) rectangle (3,-2);
				\node<1-> at (1.5,-1) {$x$};
				\draw<1-> (1,-2) -- (1,-2.2);
				\draw<1->[-triangle 90] (1,-2.5) -- (1,-2.2);
				\node<1-> at (.5,-2.25) {$1^n$};
				\draw<1->[-triangle 90] (2,-2) -- (2,-2.3);
				\draw<1-> (2,-2.3) -- (2,-2.5);
				\node<1-> at (2.5,-2.25) {$d_n$};
			%f
				\draw<1->[thick] (0,-2.5) rectangle (3,-4.5);
				\node<1-> at (1.5,-3.5) {$f$};
				\draw<1-> (1,-4.5) -- (1,-4.7);
				\draw<1->[-triangle 90] (1,-5) -- (1,-4.7);
				\node<1-> at (.5,-4.75) {$1^m$};
				\draw<1->[-triangle 90] (2,-4.5) -- (2,-4.8);
				\draw<1-> (2,-4.8) -- (2,-5);
				\node<1-> at (2.5,-4.75) {$d'_m$};
			%f(x)
				\draw<1-> (-1.5,.5) rectangle (3.5,-4.5);
				\node<1-> at (-.75,-2) {$f(x)$};
		\end{tikzpicture}
	\end{minipage}
	\hfill
	\begin{minipage}{.47\textwidth}
		\begin{definition}
			\small
			A function $f:[0,1]\to\RR$ is called computable, \newline if there is a machine that,\newline when given oracle access to approximations to the argument $x$,\newline returns approximations to the value $f(x)$.
		\end{definition}
	\end{minipage}
\end{frame}
