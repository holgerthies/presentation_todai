\section{Dynamic Systems and the Shadowing Lemma}
\subsection{Overview}
\begin{frame}[<+->]
\frametitle{Shadowing}
Consider the logistic map $f(x) = a x  (1-x)$ for $a \in \RR$ \\
The orbit of $f$ is given by the recurrence 
$y_{i+1} = f(y_i)$ and $y_0 \in \RR$.
\pause
\begin{definition}
	The sequence $(x_i)$ is called an $\alpha$-pseudo-orbit for $f$ if 
	$$
	\| x_{i+1} - f(x_i) \| < \alpha
	$$ 
	A real orbit $(y_i)$ $\beta$-shadows a pseudo-orbit $(x_i)$ if 
	$$
	\| x_{i} - y_i \| < \beta
	$$ 
\end{definition}
\pause
Question (originally considered by Hammel, Yorke and Grebogi in 1987): How long does a shadowing orbit for the logistic map exist for different parameters $a$ and $x_0$?
\end{frame}
\begin{frame}[<+->]
\frametitle{Computing the shadowing orbit}
\begin{itemize}
\item \irram can be used to exactly compute a shadowing orbit of length $N$
\item First compute the psuedo orbit $(x_i)$
\item Start with $y_N := x_N$ and iteratively compute $y_{i-1} = f^{-1}(y_i)$
\item For $f^{-1} = 0.5 \pm \sqrt{0.25-\frac{x}{a}}$ choose the value that is on the same side of $0.5$ as $x_i$
\item The backward procedure will stay close to the pseudo-orbit since $f^{-1}$ is a contraction.
\item Note that the algorithm might fail
\end{itemize}
\end{frame}